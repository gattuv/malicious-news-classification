\chapter{Introduction}

The practice of producing false narratives and spreading lies to influence the masses was seen as early as 1279 B.C.E. when Ramesses II told his people that his first major excursion, only 5 years into his reign as pharaoh, against the Hittites was a resounding success.  However, the reality was that he had barely managed to make it home alive after being duped by Hittite spies and losing a large majority of his army in his first battle.  As the head of the totalitarian government of Egypt, Ramesses II was able to literally etch his false narrative into stone in an effort to convince his subjects he was superior to their enemies and would build a legacy greater than the pharaohs before him \cite{weir}.  Even some of America's greatest heroes are guilty of deceiving the public to sway public opinion, such as Benjamin Franklin who falsely portrayed Native Americans as murderous scalping enthusiasts, working with King George III, in order to rile up revolutionaries and gain the French's support in the American Revolution \cite{politicoHistoryOfFakeNews}.

Modern technology and higher literacy rates have made it possible for more people to understand the world around them; thus, much of the modern populace can now freely form their own opinions rather than faithfully accept whatever they are told.  The impact of unchecked social media proliferation has led to a new surge of information sharing by established and freelance journalists, small digital publications, large media conglomerates, and inexperienced, yet opinionated and influential, individuals who had previously never shown an interest in public service of any kind.  Recent estimates show that over 67\% of Americans get at least some of their news from social media \cite{pewNewsAcrossSocialMedia}.  The influx of unregulated media vying for Americans' attention makes it difficult to determine whether or not the sources of information Americans now turn to are even legitimate, let alone accurate.  Furthermore, the mounting pressure many journalists face to meet urgent deadlines or publish alerts and updates quickly, in as little as 15 minutes, may make verifying a story for accuracy exceedingly difficult \cite{breakingNewsDrill}.

However, the scenario that unfolded during the United States' 2016 presidential campaign season had little to do with journalists failing to do their due diligence.  By many accounts, fake news, or stories "[intended] to deceive, often geared towards getting clicks", spread through social media platforms like Facebook and Twitter, enwrapping millions of Americans and influencing their perception of the leading contenders in the presidential race \cite{fakeNewsDef}.  In fact, the FBI, along with other American intelligence agencies, concluded that the Russian government was the real author of some of the most trending fake news, including "pizzagate" \cite{newsweekSputnik}.  More recently, British Prime Minister Theresa May publicly denounced Russia for undermining the West's free society by "weaponizing information" and "deploying [Russia's] state-run media organizations to plant fake stories" \cite{englandFakeNews}.  Similarly, Spain accused Russia for employing similar tactics to divide the Spanish people with disinformation \cite{cataloniaFakeNews}.  Thus, the threat posed by maliciously written articles still looms on many parts of the web and could impact upcoming democratic elections.

This study seeks to lay the foundations for developing an autonomous system capable of filtering out untrustworthy content, such as the material that was spread on social media by Russian bots \cite{russianBots}.  One property of a successful system would be the ability to filter out deceitful content written by repeat offenders.  For example, Paul Horner, one of the most prolific fake news authors, composed his work just to showcase how gullible the public was \cite{fakeNewsAuthor}.  In the past, forensic linguists have shown that analysis of idiolects, individuals' distinctive use of language, can help to positively identify anonymous authors, like James Madison as the author of some disputed Federalist Papers, and Ted Kaczynski as the Unabomber and author behind the terrorist manifesto titled \textit{An Industrial Society and Its Future} \cite{federalistPapers, forensicLinguistics}.  To that effect, this study aims to train classifiers that can use not only article content, but also the hidden markers indicative of authors that aim to deceive their readers.

Chapter 2 of this thesis discusses the select algorithms and techniques that are ultimately used in this study.  Afterwards, an overview of the current state of the art in discerning fact from fiction is provided in Chapter 3.  Next, the results of the various experiments using machine learning are presented in Chapter 4.  Finally, this study is concluded in Chapter 5 with an analysis of its successes and shortcomings.

